\documentclass[journal,12pt,twocolumn]{IEEEtran}

\usepackage{setspace}
\usepackage{gensymb}
\singlespacing
\usepackage[cmex10]{amsmath}

\usepackage{amsthm}

\usepackage{mathrsfs}
\usepackage{txfonts}
\usepackage{stfloats}
\usepackage{float}
\usepackage{bm}
\usepackage{cite}
\usepackage{cases}
\usepackage{subfig}

\usepackage{longtable}
\usepackage{multirow}

\usepackage{enumitem}
\usepackage{mathtools}
\usepackage{steinmetz}
\usepackage{tikz}
\usepackage{circuitikz}
\usepackage{verbatim}
\usepackage{tfrupee}
\usepackage[breaklinks=true]{hyperref}
\usepackage{graphicx}
\usepackage{tkz-euclide}

\usetikzlibrary{calc,math}
\usepackage{listings}
    \usepackage{color}                                            %%
    \usepackage{array}                                            %%
    \usepackage{longtable}                                        %%
    \usepackage{calc}                                             %%
    \usepackage{multirow}                                         %%
    \usepackage{hhline}                                           %%
    \usepackage{ifthen}                                           %%
    \usepackage{lscape}     
\usepackage{multicol}
\usepackage{chngcntr}
\usepackage{float}
\restylefloat{table}

\DeclareMathOperator*{\Res}{Res}

\renewcommand\thesection{\arabic{section}}
\renewcommand\thesubsection{\thesection.\arabic{subsection}}
\renewcommand\thesubsubsection{\thesubsection.\arabic{subsubsection}}

\renewcommand\thesectiondis{\arabic{section}}
\renewcommand\thesubsectiondis{\thesectiondis.\arabic{subsection}}
\renewcommand\thesubsubsectiondis{\thesubsectiondis.\arabic{subsubsection}}


\hyphenation{op-tical net-works semi-conduc-tor}
\def\inputGnumericTable{}                                 %%

\lstset{
%language=C,
frame=single, 
breaklines=true,
columns=fullflexible
}
\begin{document}

\newcommand{\BEQA}{\begin{eqnarray}}
\newcommand{\EEQA}{\end{eqnarray}}
\newcommand{\define}{\stackrel{\triangle}{=}}
\bibliographystyle{IEEEtran}
\raggedbottom
\setlength{\parindent}{0pt}
\providecommand{\mbf}{\mathbf}
\providecommand{\pr}[1]{\ensuremath{\Pr\left(#1\right)}}
\providecommand{\qfunc}[1]{\ensuremath{Q\left(#1\right)}}
\providecommand{\sbrak}[1]{\ensuremath{{}\left[#1\right]}}
\providecommand{\lsbrak}[1]{\ensuremath{{}\left[#1\right.}}
\providecommand{\rsbrak}[1]{\ensuremath{{}\left.#1\right]}}
\providecommand{\brak}[1]{\ensuremath{\left(#1\right)}}
\providecommand{\lbrak}[1]{\ensuremath{\left(#1\right.}}
\providecommand{\rbrak}[1]{\ensuremath{\left.#1\right)}}
\providecommand{\cbrak}[1]{\ensuremath{\left\{#1\right\}}}
\providecommand{\lcbrak}[1]{\ensuremath{\left\{#1\right.}}
\providecommand{\rcbrak}[1]{\ensuremath{\left.#1\right\}}}
\theoremstyle{remark}
\newtheorem{rem}{Remark}
\newcommand{\sgn}{\mathop{\mathrm{sgn}}}
\providecommand{\abs}[1]{\vert#1\vert}
\providecommand{\res}[1]{\Res\displaylimits_{#1}} 
\providecommand{\norm}[1]{\lVert#1\rVert}
%\providecommand{\norm}[1]{\lVert#1\rVert}
\providecommand{\mtx}[1]{\mathbf{#1}}
\providecommand{\mean}[1]{E[ #1 ]}
\providecommand{\fourier}{\overset{\mathcal{F}}{ \rightleftharpoons}}
%\providecommand{\hilbert}{\overset{\mathcal{H}}{ \rightleftharpoons}}
\providecommand{\system}{\overset{\mathcal{H}}{ \longleftrightarrow}}
	%\newcommand{\solution}[2]{\textbf{Solution:}{#1}}
\newcommand{\solution}{\noindent \textbf{Solution: }}
\newcommand{\cosec}{\,\text{cosec}\,}
\providecommand{\dec}[2]{\ensuremath{\overset{#1}{\underset{#2}{\gtrless}}}}
\newcommand{\myvec}[1]{\ensuremath{\begin{pmatrix}#1\end{pmatrix}}}
\newcommand{\mydet}[1]{\ensuremath{\begin{vmatrix}#1\end{vmatrix}}}
\numberwithin{equation}{subsection}
\makeatletter
\@addtoreset{figure}{problem}
\makeatother
\let\StandardTheFigure\thefigure
\let\vec\mathbf
\renewcommand{\thefigure}{\theproblem}
\def\putbox#1#2#3{\makebox[0in][l]{\makebox[#1][l]{}\raisebox{\baselineskip}[0in][0in]{\raisebox{#2}[0in][0in]{#3}}}}
     \def\rightbox#1{\makebox[0in][r]{#1}}
     \def\centbox#1{\makebox[0in]{#1}}
     \def\topbox#1{\raisebox{-\baselineskip}[0in][0in]{#1}}
     \def\midbox#1{\raisebox{-0.5\baselineskip}[0in][0in]{#1}}
\vspace{3cm}
\title{AI1103--Assignment-3}
\author{Name: Aravinda Kumar Reddy Thippareddy\\Roll.No.:CS20BTECH11053}
\maketitle
\newpage
\bigskip
\renewcommand{\thefigure}{\theenumi}
\renewcommand{\thetable}{\theenumi}
Download all latex-tikz codes from
%
\begin{lstlisting}
https://github.com/AravindCSEiith/Probability-and-Random-variables_AI1103_Asignment-3/blob/main/Assignment-3--AI1103.tex
\end{lstlisting}
\section*{Question}
Prove that for $r=1,2,3,...,n$
\begin{align}
    \frac{1}{\Gamma{\brac}(r{\brac})}\int_{\mu}^{\infty}{t^{r-1}}{e^{-t}}dt=\sum_{x=0}^{r-1}\frac{{e^{-\mu}}{{\mu}^x}}{x!}
\end{align}
\section*{Solution}
The gamma function of a positive integer 'm' is given by,
\begin{align}
    \Gamma(m)=(m-1)!
\end{align}
\subsection*{Integration By Parts(IBP):}
This is a special method of integration used for integrating product of two functions. Consider two real valued integrable functions 'u' and 'dv'. Now using Integration By Parts we can write;
\begin{align}
    \int{u}{dv}=uv-\int{v}{du}
\end{align}
Applying IBP to the L.H.S of the equation 0.0.1;
\begin{align}
     L.H.S&=\frac{1}{\Gamma{\brac}(r{\brac})}\int_{\mu}^{\infty}{t^{r-1}}{e^{-t}}dt\\
     &=\frac{1}{\Gamma{\brac}(r{\brac})}\int_{\mu}^{\infty}{t^{r-1}}d({-e^{-t}})\\
     &=\frac{1}{\Gamma{\brac}(r{\brac})}\left({t^{r-1}}(-{e^{-t}})|_{\mu}^{\infty}-\int_{\mu}^{\infty}\left(-{e^{-t}}\right)(r-1){t^{r-2}}dt\right)\\
     &=\frac{1}{(r-1)!}\left({e^{-\mu}}{\mu^{r-1}}\right)+\frac{1}{(r-2)!}\left(\int_{\mu}^{\infty}{t^{r-2}}d(-{e^{-t}})\right)
\end{align}
Similarly if we go on applying IBP, we will get;
\begin{align}
   L.H.S &=\frac{\left({e^{-\mu}}{\mu^{r-1}}\right)}{(r-1)!}+\frac{\left({e^{-\mu}}{\mu^{r-1}}\right)}{(r-2)!}+..+\frac{1}{(1)!}\left(\int_{\mu}^{\infty}{t^{1}}d(-{e^{-t}})\right)\\
   &=\frac{\left({e^{-\mu}}{\mu^{r-1}}\right)}{(r-1)!}+\frac{\left({e^{-\mu}}{\mu^{r-1}}\right)}{(r-2)!}+..+\frac{\left({e^{-\mu}}{\mu^1}\right)}{(1)!}+\frac{\left({e^{-\mu}}{\mu^0}\right)}{(0)!}\\
   &=\sum_{x=0}^{r-1}\frac{{e^{-\mu}}{{\mu}^x}}{x!}=R.H.S
\end{align}
Hence, it is proved.\\
\fbox{\begin{minipage}{20em}
Therefore, for $r=1,2,3,...,n$
\begin{align}
    \frac{1}{\Gamma{\brac}(r{\brac})}\int_{\mu}^{\infty}{t^{r-1}}{e^{-t}}dt=\sum_{x=0}^{r-1}\frac{{e^{-\mu}}{{\mu}^x}}{x!}
\end{align}
\end{minipage}}
\end{document}