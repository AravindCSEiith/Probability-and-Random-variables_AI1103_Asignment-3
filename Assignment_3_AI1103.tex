\documentclass[journal,12pt,twocolumn]{IEEEtran}

\usepackage{setspace}
\usepackage{gensymb}
\singlespacing
\usepackage[cmex10]{amsmath}

\usepackage{amsthm}

\usepackage{mathrsfs}
\usepackage{txfonts}
\usepackage{stfloats}
\usepackage{float}
\usepackage{bm}
\usepackage{cite}
\usepackage{cases}
\usepackage{subfig}

\usepackage{longtable}
\usepackage{multirow}

\usepackage{enumitem}
\usepackage{mathtools}
\usepackage{steinmetz}
\usepackage{tikz}
\usepackage{circuitikz}
\usepackage{verbatim}
\usepackage{tfrupee}
\usepackage[breaklinks=true]{hyperref}
\usepackage{graphicx}
\usepackage{tkz-euclide}

\usetikzlibrary{calc,math}
\usepackage{listings}
    \usepackage{color}                                            %%
    \usepackage{array}                                            %%
    \usepackage{longtable}                                        %%
    \usepackage{calc}                                             %%
    \usepackage{multirow}                                         %%
    \usepackage{hhline}                                           %%
    \usepackage{ifthen}                                           %%
    \usepackage{lscape}     
\usepackage{multicol}
\usepackage{chngcntr}
\usepackage{float}
\restylefloat{table}

\DeclareMathOperator*{\Res}{Res}

\renewcommand\thesection{\arabic{section}}
\renewcommand\thesubsection{\thesection.\arabic{subsection}}
\renewcommand\thesubsubsection{\thesubsection.\arabic{subsubsection}}

\renewcommand\thesectiondis{\arabic{section}}
\renewcommand\thesubsectiondis{\thesectiondis.\arabic{subsection}}
\renewcommand\thesubsubsectiondis{\thesubsectiondis.\arabic{subsubsection}}


\hyphenation{op-tical net-works semi-conduc-tor}
\def\inputGnumericTable{}                                 %%

\lstset{
%language=C,
frame=single, 
breaklines=true,
columns=fullflexible
}
\begin{document}

\newcommand{\BEQA}{\begin{eqnarray}}
\newcommand{\EEQA}{\end{eqnarray}}
\newcommand{\define}{\stackrel{\triangle}{=}}
\bibliographystyle{IEEEtran}
\raggedbottom
\setlength{\parindent}{0pt}
\providecommand{\mbf}{\mathbf}
\providecommand{\pr}[1]{\ensuremath{\Pr\left(#1\right)}}
\providecommand{\qfunc}[1]{\ensuremath{Q\left(#1\right)}}
\providecommand{\sbrak}[1]{\ensuremath{{}\left[#1\right]}}
\providecommand{\lsbrak}[1]{\ensuremath{{}\left[#1\right.}}
\providecommand{\rsbrak}[1]{\ensuremath{{}\left.#1\right]}}
\providecommand{\brak}[1]{\ensuremath{\left(#1\right)}}
\providecommand{\lbrak}[1]{\ensuremath{\left(#1\right.}}
\providecommand{\rbrak}[1]{\ensuremath{\left.#1\right)}}
\providecommand{\cbrak}[1]{\ensuremath{\left\{#1\right\}}}
\providecommand{\lcbrak}[1]{\ensuremath{\left\{#1\right.}}
\providecommand{\rcbrak}[1]{\ensuremath{\left.#1\right\}}}
\theoremstyle{remark}
\newtheorem{rem}{Remark}
\newcommand{\sgn}{\mathop{\mathrm{sgn}}}
\providecommand{\abs}[1]{\vert#1\vert}
\providecommand{\res}[1]{\Res\displaylimits_{#1}} 
\providecommand{\norm}[1]{\lVert#1\rVert}
%\providecommand{\norm}[1]{\lVert#1\rVert}
\providecommand{\mtx}[1]{\mathbf{#1}}
\providecommand{\mean}[1]{E[ #1 ]}
\providecommand{\fourier}{\overset{\mathcal{F}}{ \rightleftharpoons}}
%\providecommand{\hilbert}{\overset{\mathcal{H}}{ \rightleftharpoons}}
\providecommand{\system}{\overset{\mathcal{H}}{ \longleftrightarrow}}
	%\newcommand{\solution}[2]{\textbf{Solution:}{#1}}
\newcommand{\solution}{\noindent \textbf{Solution: }}
\newcommand{\cosec}{\,\text{cosec}\,}
\providecommand{\dec}[2]{\ensuremath{\overset{#1}{\underset{#2}{\gtrless}}}}
\newcommand{\myvec}[1]{\ensuremath{\begin{pmatrix}#1\end{pmatrix}}}
\newcommand{\mydet}[1]{\ensuremath{\begin{vmatrix}#1\end{vmatrix}}}
\numberwithin{equation}{subsection}
\makeatletter
\@addtoreset{figure}{problem}
\makeatother
\let\StandardTheFigure\thefigure
\let\vec\mathbf
\renewcommand{\thefigure}{\theproblem}
\def\putbox#1#2#3{\makebox[0in][l]{\makebox[#1][l]{}\raisebox{\baselineskip}[0in][0in]{\raisebox{#2}[0in][0in]{#3}}}}
     \def\rightbox#1{\makebox[0in][r]{#1}}
     \def\centbox#1{\makebox[0in]{#1}}
     \def\topbox#1{\raisebox{-\baselineskip}[0in][0in]{#1}}
     \def\midbox#1{\raisebox{-0.5\baselineskip}[0in][0in]{#1}}
\vspace{3cm}
\title{AI1103--Assignment-3}
\author{Name: Aravinda Kumar Reddy Thippareddy\\Roll.No.:CS20BTECH11053}
\maketitle
\newpage
\bigskip
\renewcommand{\thefigure}{\theenumi}
\renewcommand{\thetable}{\theenumi}
Download all python codes from 
%
\begin{lstlisting}
https://github.com/AravindCSEiith/Probability-and-Random-variables_AI1103_Asignment-3/blob/main/ASSIGNMENT_3_AI1103.py
\end{lstlisting}
and latex-tikz codes from
%
\begin{lstlisting}
https://github.com/AravindCSEiith/Probability-and-Random-variables_AI1103_Asignment-3/blob/main/Assignment_3_AI1103.tex
\end{lstlisting}
\section*{Question}
Let the random variable X have the distribution $P(X=0)=P(X=3)=p$, $P(X=1)=1-3p$ for $0{\leq}p{\leq}\frac{1}{2}$. What is the maximum value of V(X)?
\begin{enumerate}[label=\Alph*)]
    \item 3
    \item 4
    \item 5
    \item 6
    \item none
\end{enumerate}
\section*{Solution}
\begin{table}[h]
    \centering

       \caption{Distribution of 'X'}

    \begin{tabular}{|c|c|c|c|c|}
    \hline
        Value of X & 0 & 1 & 3 & k\\
        \hline
        Probability of 'X'& p & $1-3p$ & p & ?\\
        
        \hline
    \end{tabular}
\end{table}
We know that "sum of all probabilities$=1$". Hence,
\begin{align}
    P(X=0)+P(X=1)+P(X=3)+P(X=k) &=1 \\
    (p)+(1-3p)+(p)+P(X=k) &= 1 \\
    1-p+P(X=k) &= 1\\
    P(X=k) &=p
\end{align}
The expectation value of 'X', E(X) is,
\begin{align}
    E(X)={\bar{X}} &={\sum}x_i.P(X=x_i)\\
    &=(0)(p)+(1)(1-3p)+(3)(p)+(k)(p)\\
    &=1+kp
\end{align}
Now the expression for variance,
\begin{align}
    V(X)&={\sum}P(X=x_i)(x_i-\bar{X})^2\\
    V(X)&={\sum}P(X=x_i)(x_i-(1+kp))^2\\
\end{align}
Let $V(X)=f(k)$.
\begin{align}
    V(X)&=f(k)=(p-p^2)k^2+(-2p)k+(6p)\\
    V(X)&=f(k)=(p-p^2)k^2-2pk+6p
\end{align}
Compare the above quadratic equation with the general standard form of quadratic equation $'ax^2+bx+c'$. We get,
\begin{align}
    a&=p-p^2\\
    b&=-2p\\
    c&=6p
\end{align}
$a>0$, $\forall p{\in}(0,1)$. It is given in question that $0{\leq}p{\leq}\frac{1}{2}$. Hence $a{\geq}0$. Hence maximum value of quadratic expression, V(X) is ${+\infty}$, as $k{\rightarrow}{\pm}{\infty}$.
\begin{figure}[H]
\centering
\includegraphics[width=\columnwidth]{Screenshot_20210415-224917.png}

\end{figure}
Hence maximum value of V(X) is +\infty.\\
\rightline{Answer : Option E}\\
\fbox{\begin{minipage}{16em}
\centering
 Maximum value of V(X)=+\infty
\end{minipage}}



\end{document}
